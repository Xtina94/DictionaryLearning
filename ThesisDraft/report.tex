\documentclass[11pt,a4paper,titlepage]{article}
\usepackage[a4paper]{geometry}
\usepackage[utf8]{inputenc}
\usepackage[english]{babel}
\usepackage{lipsum}

\usepackage{amsmath, amssymb, amsfonts, amsthm, mathtools}
% mathtools for: Aboxed (put box on last equation in align envirenment)
\usepackage{microtype} %improves the spacing between words and letters

\usepackage{lipsum}
\usepackage{threeparttable}
\usepackage{tabularx}
\usepackage{multirow}
\usepackage{booktabs}
\newcommand{\tabitem}{~~\llap{\textbullet}~~}
\usepackage{graphicx}
\graphicspath{ {./pics/} {./eps/}}
\usepackage{epsfig}
\usepackage{epstopdf}

%%%%%%%%%%%%%%%%%%%%%%%%%%%%%%%%%%%%%%%%%%%%%%%%%%
%% COLOR DEFINITIONS
%%%%%%%%%%%%%%%%%%%%%%%%%%%%%%%%%%%%%%%%%%%%%%%%%%
\usepackage[dvipsnames]{xcolor} % Enabling mixing colors and color's call by 'svgnames'
%%%%%%%%%%%%%%%%%%%%%%%%%%%%%%%%%%%%%%%%%%%%%%%%%%
\definecolor{MyColor1}{rgb}{0.2,0.4,0.6} %mix personal color
\newcommand{\textb}{\color{Black} \usefont{OT1}{lmss}{m}{n}}
\newcommand{\blue}{\color{MyColor1} \usefont{OT1}{lmss}{m}{n}}
\newcommand{\blueb}{\color{MyColor1} \usefont{OT1}{lmss}{b}{n}}
\newcommand{\red}{\color{LightCoral} \usefont{OT1}{lmss}{m}{n}}
\newcommand{\green}{\color{Turquoise} \usefont{OT1}{lmss}{m}{n}}
%%%%%%%%%%%%%%%%%%%%%%%%%%%%%%%%%%%%%%%%%%%%%%%%%%


%%%%%%%%%%%%%%%%%%%%%%%%%%%%%%%%%%%%%%%%%%%%%%%%%%
%% FONTS AND COLORS
%%%%%%%%%%%%%%%%%%%%%%%%%%%%%%%%%%%%%%%%%%%%%%%%%%
%    SECTIONS
%%%%%%%%%%%%%%%%%%%%%%%%%%%%%%%%%%%%%%%%%%%%%%%%%%
\usepackage{titlesec}
\usepackage{sectsty}
%%%%%%%%%%%%%%%%%%%%%%%%
%set section/subsections HEADINGS font and color
\sectionfont{\color{MyColor1}}  % sets colour of sections
\subsectionfont{\color{MyColor1}}  % sets colour of sections

%set section enumerator to arabic number (see footnotes markings alternatives)
\renewcommand\thesection{\arabic{section}.} %define sections numbering
\renewcommand\thesubsection{\thesection\arabic{subsection}} %subsec.num.

%define new section style
\newcommand{\mysection}{
\titleformat{\section} [runin] {\usefont{OT1}{lmss}{b}{n}\color{MyColor1}}
{\thesection} {3pt} {} }

%%%%%%%%%%%%%%%%%%%%%%%%%%%%%%%%%%%%%%%%%%%%%%%%%%
%		CAPTIONS
%%%%%%%%%%%%%%%%%%%%%%%%%%%%%%%%%%%%%%%%%%%%%%%%%%
\usepackage{caption}
\usepackage{subcaption}
%%%%%%%%%%%%%%%%%%%%%%%%
\captionsetup[figure]{labelfont={color=Turquoise}}

%%%%%%%%%%%%%%%%%%%%%%%%%%%%%%%%%%%%%%%%%%%%%%%%%%
%		!!!EQUATION (ARRAY) --> USING ALIGN INSTEAD
%%%%%%%%%%%%%%%%%%%%%%%%%%%%%%%%%%%%%%%%%%%%%%%%%%
%using amsmath package to redefine eq. numeration (1.1, 1.2, ...)
%%%%%%%%%%%%%%%%%%%%%%%%
\renewcommand{\theequation}{\thesection\arabic{equation}}

%set box background to grey in align environment
\usepackage{etoolbox}% http://ctan.org/pkg/etoolbox
\makeatletter
\patchcmd{\@Aboxed}{\boxed{#1#2}}{\colorbox{black!15}{$#1#2$}}{}{}%
\patchcmd{\@boxed}{\boxed{#1#2}}{\colorbox{black!15}{$#1#2$}}{}{}%
\makeatother
%%%%%%%%%%%%%%%%%%%%%%%%%%%%%%%%%%%%%%%%%%%%%%%%%%



\makeatletter
\let\reftagform@=\tagform@
\def\tagform@#1{\maketag@@@{(\ignorespaces\textcolor{red}{#1}\unskip\@@italiccorr)}}
\renewcommand{\eqref}[1]{\textup{\reftagform@{\ref{#1}}}}
\makeatother
\usepackage{hyperref}
\hypersetup{colorlinks=true}

%% LISTS CONFIGURATION %%
\usepackage{enumitem}
\setlist[enumerate,1]{start=0}
\renewcommand{\labelenumii}{\theenumii}
\renewcommand{\theenumii}{\theenumi.\arabic{enumii}.}

%%%%%%%%%%%%%%%%%%%%%%%%%%%%%%%%%%%%%%%%%%%%%%%%%%
%% PREPARE TITLE
%%%%%%%%%%%%%%%%%%%%%%%%%%%%%%%%%%%%%%%%%%%%%%%%%%
\title{\blue Satellite Communications \\
\blueb Project presentation}
\author{Ana Reviejo Jiménez \\ Marta Munilla Díez\\ Oscar Pla Terrada\\ Davide Peron\\ Cristina Gava\\ Javier Garcia Camin}
\date{\today}
%%%%%%%%%%%%%%%%%%%%%%%%%%%%%%%%%%%%%%%%%%%%%%%%%%




% A diagram of TeX software
% Author: Stefan Kottwitz
% https://www.packtpub.com/hardware-and-creative/latex-cookbook
%\documentclass[border=10pt]{standalone} 
%%%<
\usepackage{verbatim}
%%%>
\begin{comment}
:Title: A circular diagram of a TeX workflow
:Tags: Diagrams;Smartdiagram;Cookbook
:Author: Stefan Kottwitz
:Slug: smart-constellation

A diagram presenting TeX software using the
smartdiagram package.
\end{comment}
\usepackage{smartdiagram}

\begin{document}
\maketitle

\section{Team presentation}{%
\begin{center}
\begin{minipage}{0.2\textwidth}
	\includegraphics[width=\linewidth]{Javier.jpg}
\end{minipage}\hspace{0.5cm}
\begin{minipage}{0.7\textwidth}
	\textbf{Javier García Camín}

	Polytechnic University of Madrid (UPM)

In this project he is the Project supervisor and deals with the orbit and constellation analysis.
\end{minipage}\\ \vspace{0.5cm}
\begin{minipage}{0.2\textwidth}
		\includegraphics[width=\linewidth]{Oscar.png}
\end{minipage}\hspace{0.5cm}
\begin{minipage}{0.7\textwidth}
	\textbf{Óscar Pla Terrada}

	Polytechnic University of Valencia (UPV)

In this project he deals with the Costs Management and studies the possibility to exploit currently used technologies.
\end{minipage}\\ \vspace{0.5cm}
\begin{minipage}{0.2\textwidth}
	\includegraphics[width=\linewidth]{Cri.png}
\end{minipage}\hspace{0.5cm}
\begin{minipage}{0.7\textwidth}
	\textbf{Cristina Gava} received the B.Sc. degree in Information Engineering from the University of Padova, Italy, in 2016, and is currently pursuing the M.Sc. degree in Telecommunication Engineering. She is spending a period abroad as an exchange student at Polytechnic University of Madrid (UPM).

	In this project she deals with the physical layer of the satellite communication module and the Bit Error Control.
\end{minipage}\\ \vspace{0.5cm}
\begin{minipage}{0.2\textwidth}
		\includegraphics[width=\linewidth]{Marta.jpg}
\end{minipage}\hspace{0.5cm}
\begin{minipage}{0.7\textwidth}
	\textbf{Marta Munilla Diez}

University of Cantabria (UC)

In this project she deals with the Platform requirements.
\end{minipage}\\ \vspace{0.5cm}
\begin{minipage}{0.2\textwidth}
	\includegraphics[width=\linewidth]{Davide.jpg}
\end{minipage}\hspace{0.5cm}
\begin{minipage}{0.7\textwidth}
\textbf{Davide Peron} received the B.Sc. degree in Information Engineering from the University of Padova, Italy, in 2016, and is currently pursuing the M.Sc. degree in Telecommunication Engineering. He is spending a period abroad as an exchange student at Polytechnic University of Madrid (UPM).

In this project he deals with the risks management and the network and MAC layers of the satellite communication module.
\end{minipage}\\ \vspace{0.5cm}
\begin{minipage}{0.2\textwidth}
		\includegraphics[width=\linewidth]{Ana.png}
\end{minipage}\hspace{0.5cm}
\begin{minipage}{0.7\textwidth}
	\textbf{Ana María Reviejo Jiménez}

	Polytechnic University of Madrid (UPM)

In this project she deals with Ground Segment requirements and the design of a proper system architecture.
\end{minipage}\\
\end{center}

	\begin{table}[ht]
	\small
	\centering
	\renewcommand{\arraystretch}{1}% Tighter
	\begin{tabular}{@{}lll@{}}
		\toprule
			Name & Role & Responsibilities\\
		\midrule
			Javier García Camín & Project Leader (PoC) &
			\parbox{7cm}{
				\begin{itemize}
				\item Project supervisor
				\item Choosing two solutions for satellite orbits and constellations taking in account the regions to cover and the type of service.
				\end{itemize}
			}\\ \hline
			Oscar Pla Terrada &  &
			\parbox{7cm}{
				\begin{itemize}
				\item Costs Management
				\item  Looking for information about already known satellite technologies in this area.
				\end{itemize}
			}\\ \hline
			Ana María Reviejo Jiménez &  &
			\parbox{7cm}{
				\begin{itemize}
				\item Description of Ground Segment Requirements
				\item Study of a solution for system architecture.
				\end{itemize}
			}\\ \hline
			Davide Peron &  &
			\parbox{7cm}{
				\begin{itemize}
				\item Risks Management
				\item Description of the satellite requirements.
				\end{itemize}
			}\\ \hline
			Cristina Gava &  &
			\parbox{7cm}{
				\begin{itemize}
				\item Description of the physical layer.
				\end{itemize}
			}\\ \hline
			Marta Munilla Diez &  &
			\parbox{7cm}{
				\begin{itemize}
				\item Description of the platform requirements.
				\end{itemize}
			}\\
		\bottomrule
	\end{tabular}
		\caption{Role and responsibilities of each component of the team.}
		\label{table:roles}
	\end{table}
}

\section{Problem description}{

\begin{figure}[b]
	\centering
	\includegraphics[width=0.4\linewidth]{System.png}
	\caption{Simple system representation}
	\label{fig:system}
	\end{figure}

	This project results from the necessity of having a good broadband coverage of polar
	areas and the land areas of Northern Europe and Russia: this means the coverage of
	latitudes over the 60 deg. The subjects interested in this kind of communication are mostly
	industries involved in economic sector: they need a reliable communication system able to
	provide a service of 50 Mbps in download and 5 Mbps in upload.

	The aim is to project a system able to provide a continuous, reliable and feasible
	communication service, maximizing the number of users allowed to access it over 60 deg.
	latitudes and minimizing the costs.

	To do that, services in narrowband communication using LEO satellites are not useful,
	since the broadband communication required is not feasible with this technology.\\

	A simple representation of the system to be built is shown in figure \ref{fig:system}: after a collective analysis, we came to 	the conclusion that no base station is required in this case, since the major issue is to provide wireless connection 				among users.

}\label{sec:description}

\section{List of tasks}{
	\begin{enumerate}
		\item Plan of the project studying the problem and brainstorming of the possible solutions
		\item Choose two solutions for satellite orbits and constellations taking in account the regions to cover and the type of service
		\item Look for information about satellite technologies that are already in use in these areas
		\item Study a solution for system architecture
		\item Description of the satellite requirements
		\begin{enumerate}
			\item Selection of carrier frequencies
			\item Selection of MODCOD
			\item Media Access Control and multiplexing (TDMA, FDMA, CDMA)
		\end{enumerate}
		\item Description of the physical layer
		\begin{enumerate}
			\item Selection of Antenna system (MIMO or SIMO, type of antenna)
			\item Specification about transponder (number and bandwidth)
			\item Bit Error Control
		\end{enumerate}
		\item Description of the platform Requirements
		\begin{enumerate}
			\item Type of platform to ensure a good Attitude Control
			\item Electrical Power range estimation
		\end{enumerate}
		\item Description of Ground Segment Requirements
		\begin{enumerate}
			\item Action allowed to the different kind of users
			\item User's antenna specifications
			\item Requirements of Uplink and Control stations
		\end{enumerate}
		\item Costs Management
		\item Risks Management
	\end{enumerate}
}\label{sec:tasks}

\section{GANTT Chart}{
	In the follow are proposed a timetable with the different tasks and the correspondent GANTT chart.

	\vspace{1cm}
	\begin{minipage}{0.48\textwidth}
		\centering
			\includegraphics[width=\linewidth]{Tasks1.jpg}
			\label{fig:tasks1}
	\end{minipage}
	\begin{minipage}{0.48\textwidth}
		\centering
			\includegraphics[width=\linewidth]{Tasks2.jpg}
			\label{fig:tasks2}
	\end{minipage}\\
	\begin{minipage}{0.48\textwidth}
		\centering
			\includegraphics[width=\linewidth]{Gantt1.jpg}
			\label{fig:gantt1}
	\end{minipage}
	\begin{minipage}{0.48\textwidth}
		\centering
			\includegraphics[width=\linewidth]{Gantt2.jpg}
			\label{fig:gantt2}
	\end{minipage}

}\label{sec:gantt}

\section{References}{
	\begin{description}
		\item [http://www.satbeams.com/] Website containing specifications and footprint of satellites already in use, can be useful to check for a solution that includes the use of technologies already used.
		\item [Comunicaciones por Satélite] Book that will be used to find useful formulas and concepts
		\item [http://ieeexplore.ieee.org] Can be used to analyse some innovative solutions or to compare our original solution with a state-of-art one.
	\end{description}
}

\section{Tools}{
	\begin{description}
		\item [Matlab] Used for simulations about the orbit, calculation about bit error rate, modulations and carrier frequencies.
		\item [System ToolKit] Used for orbit simulations.
		\item [Microsoft Office Excel] Used for calculations of link budgets.
	\end{description}
}

\section{Simulation architecture}{
The simulation architecture requires to estimate the link budget so as to understand how valuable the connection is. Its estimation can be done through a simulation script based on the main physical parameters that describe the distance between the TX and RX antennas: range, elevation/azimuth, pointing requirements, and so on.
Another script can be created to estimate the previous parameters, together with orbit parameters that allow coverage and visibility analysis. This script can take as inputs the service area and the satellite ephemeris and obtain the results through Kepler’s laws and geometric transformations to estimate coordinates in the right reference system.\\
The diagram below summarizes schematically the supposed structure of the script.

%\begin{figure}[h]
\begin{center}
\smartdiagramset{
	border color = gray,
	set color list = {Aquamarine,Turquoise,ProcessBlue,CornflowerBlue,RoyalBlue},
	back arrow disabled = true
}
\smartdiagram[flow diagram:horizontal]{Service area and the satellite ephemeris as inputs,
  Kepler's Laws \& geometric Transf., {Range, Elevation, Azimuth},Link Budget, Connection reliability}
%\caption{flow diagram for the script}
\label{fig:flow}
\end{center}
%\end{figure}
}

\section{Main difficulties along the project}{
	\begin{description}
		\item [Understanding the problem] The problem can be interpreted in several ways, we have to choose the right one and think about a proper system architecture to solve it.
		\item [Planning the correct approach] We have to design the right system architecture in order to proceed with all the subsequent steps.
		\item [Find simulation tools that fit our needs] There are many simulation tools, but not all of them fit in the right way and efficiently our needs.

		We will surely use Matlab, but we have to check for its limitations in Satellite communication field, and bridge this gap using other tools.
		\item [Work in group, languages and different academic backgrounds] We come from 4 different Universities, in different country, and we have a very different academic background.

		We have to share our knowledge taking the best of each component of the team and using a common language.
	\end{description}
}

\end{document}
