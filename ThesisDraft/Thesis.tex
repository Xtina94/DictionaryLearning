\documentclass[11pt,a4paper,titlepage]{article}
\usepackage[a4paper]{geometry}
\usepackage[utf8]{inputenc}
\usepackage[english]{babel}

\usepackage[light]{CormorantGaramond}

% \usepackage[light,math]{kurier}

\usepackage[T1]{fontenc}
\linespread{1.5}

\usepackage{amsmath, amssymb, amsfonts, amsthm, mathtools}
% mathtools for: Aboxed (put box on last equation in align envirenment)
\usepackage{microtype} %improves the spacing between words and letters

\usepackage{lipsum}
\usepackage{threeparttable}
\usepackage{tabularx}
\usepackage{multirow}
\usepackage{booktabs}
\newcommand{\tabitem}{~~\llap{\textbullet}~~}
\usepackage{graphicx}
\graphicspath{ {./pics/} {./eps/}}
\usepackage{epsfig}
\usepackage{epstopdf}

%%%%%%%%%%%%%%%%%%%%%%%%%%%%%%%%%%%%%%%%%%%%%%%%%%
%% COLOR DEFINITIONS
%%%%%%%%%%%%%%%%%%%%%%%%%%%%%%%%%%%%%%%%%%%%%%%%%%
\usepackage[dvipsnames]{xcolor} % Enabling mixing colors and color's call by 'svgnames'
%%%%%%%%%%%%%%%%%%%%%%%%%%%%%%%%%%%%%%%%%%%%%%%%%%
\definecolor{MyColor1}{rgb}{0.2,0.4,0.6} %mix personal color
\newcommand{\textb}{\color{Black} \usefont{OT1}{lmss}{m}{n}}
\newcommand{\blue}{\color{MyColor1} \usefont{OT1}{lmss}{m}{n}}
\newcommand{\blueb}{\color{MyColor1} \usefont{OT1}{lmss}{b}{n}}
\newcommand{\red}{\color{LightCoral} \usefont{OT1}{lmss}{m}{n}}
\newcommand{\green}{\color{Turquoise} \usefont{OT1}{lmss}{m}{n}}
%%%%%%%%%%%%%%%%%%%%%%%%%%%%%%%%%%%%%%%%%%%%%%%%%%

%%%%%%%%%%%%%%%%%%%%%%%%%%%%%%%%%%%%%%%%%%%%%%%%%%
%% FONTS AND COLORS
%%%%%%%%%%%%%%%%%%%%%%%%%%%%%%%%%%%%%%%%%%%%%%%%%%
%    SECTIONS
%%%%%%%%%%%%%%%%%%%%%%%%%%%%%%%%%%%%%%%%%%%%%%%%%%
\usepackage{titlesec}
\usepackage{sectsty}
%%%%%%%%%%%%%%%%%%%%%%%%
%set section/subsections HEADINGS font and color
\sectionfont{\color{MyColor1}}  % sets colour of sections
\subsectionfont{\color{MyColor1}}  % sets colour of sections

%set section enumerator to arabic number (see footnotes markings alternatives)
\renewcommand\thesection{\arabic{section}.} %define sections numbering
\renewcommand\thesubsection{\thesection\arabic{subsection}} %subsec.num.

%define new section style
\newcommand{\mysection}{
\titleformat{\section} [runin] {\usefont{OT1}{lmss}{b}{n}\color{MyColor1}}
{\thesection} {3pt} {} }

%%%%%%%%%%%%%%%%%%%%%%%%%%%%%%%%%%%%%%%%%%%%%%%%%%
%		CAPTIONS
%%%%%%%%%%%%%%%%%%%%%%%%%%%%%%%%%%%%%%%%%%%%%%%%%%
\usepackage{caption}
\usepackage{subcaption}
%%%%%%%%%%%%%%%%%%%%%%%%
\captionsetup[figure]{labelfont={color=Turquoise}}

%%%%%%%%%%%%%%%%%%%%%%%%%%%%%%%%%%%%%%%%%%%%%%%%%%
%		!!!EQUATION (ARRAY) --> USING ALIGN INSTEAD
%%%%%%%%%%%%%%%%%%%%%%%%%%%%%%%%%%%%%%%%%%%%%%%%%%
%using amsmath package to redefine eq. numeration (1.1, 1.2, ...)
%%%%%%%%%%%%%%%%%%%%%%%%
\renewcommand{\theequation}{\thesection\arabic{equation}}

%set box background to grey in align environment
\usepackage{etoolbox}% http://ctan.org/pkg/etoolbox
\makeatletter
\patchcmd{\@Aboxed}{\boxed{#1#2}}{\colorbox{black!15}{$#1#2$}}{}{}%
\patchcmd{\@boxed}{\boxed{#1#2}}{\colorbox{black!15}{$#1#2$}}{}{}%
\makeatother
%%%%%%%%%%%%%%%%%%%%%%%%%%%%%%%%%%%%%%%%%%%%%%%%%%

\makeatletter
\let\reftagform@=\tagform@
\def\tagform@#1{\maketag@@@{(\ignorespaces\textcolor{red}{#1}\unskip\@@italiccorr)}}
\renewcommand{\eqref}[1]{\textup{\reftagform@{\ref{#1}}}}
\makeatother
\usepackage{hyperref}
\hypersetup{colorlinks = true, linkcolor  = black}

%% LISTS CONFIGURATION %%
\usepackage{enumitem}
\setlist[enumerate,1]{start=0}
\renewcommand{\labelenumii}{\theenumii}
\renewcommand{\theenumii}{\theenumi.\arabic{enumii}.}
\newcommand{\cri}[1]{\textcolor{green}{\textbf{(Cri says: #1)}}}

%%%%%%%%%%%%%%%%%%%%%%%%%%%%%%%%%%%%%%%%%%%%%%%%%%
%% PREPARE TITLE
%%%%%%%%%%%%%%%%%%%%%%%%%%%%%%%%%%%%%%%%%%%%%%%%%%
\title{\blue Master Project title \\
\blueb Final master project}
\author{Cristina Gava}
\date{\today}
%%%%%%%%%%%%%%%%%%%%%%%%%%%%%%%%%%%%%%%%%%%%%%%%%%

% A diagram of TeX software
% Author: Stefan Kottwitz
% https://www.packtpub.com/hardware-and-creative/latex-cookbook
%\documentclass[border=10pt]{standalone}
%%%<
\usepackage{verbatim}
%%%>
\begin{comment}
:Title: A circular diagram of a TeX workflow
:Tags: Diagrams;Smartdiagram;Cookbook
:Author: Stefan Kottwitz
:Slug: smart-constellation

A diagram presenting TeX software using the
smartdiagram package.
\end{comment}
\usepackage{smartdiagram}

\begin{document}
\maketitle

\tableofcontents

\section{Introduction: the emerging field of graph signal processing}
\cri{Forse potresti aggiungere un'introduzione un po' più poetica su come al giorno d'oggi siamo circondati da dati in ogni angolo}\\
One of the central entities we will use in this work is the concept of \textit{graph signal}. Graphs are generic data representation structures, which are useful in a great variety of fields and applications for everyday life and technology in general. These structures are composed by two main elements: the nodes and the edges; the former are a set of points, identifying an aspect of the graph structure itself, while the latter are connections between the nodes. Several structures we can encounter in natural entities and abstract constructions can be represented by a graph structure, and when we associate values to their set of nodes and edges we obtain a graph signal. To be specific, a graph signal is seen as a finite collection of samples located at each node in the structure and which are interconnected among themselves through the edges, to which we can associate numerical values representing the weights of the connections. The metric for these weights is not unique, as it depends on which relation between the nodes we are looking at: a typical metric for graph weights may be, for example, inversely proportional to the distance (physical or not) between two different nodes, but other options are possible.

As previously mentioned, graph structures appear to be significantly helpful when they are used to represent signals, their applications are the most varied and can go from \cri{finisci la frase con degli esempi tratti dai paper}

\subsection{Representing a graph: graph learning techniques and dictionary representation}
\lipsum[1-15]

\subsection{The smoothness assumption}
\lipsum[1-15]

\section{Brief state of the art}
\lipsum[1-15]

\section{The algorithms we started from: Dorina's and Hermina's}
\lipsum[1-15]

\section{Problem presentation}
\lipsum[1-15]

\section{main improvements}
\lipsum[1-15]

\subsection{first part: the dictionary learning under smoothness constraints}
\lipsum[1-15]

\subsection{second part: the graph learning part}
\lipsum[1-15]

\subsection{Merging the two approaches}
\lipsum[1-15]

\section{further improvements}
\lipsum[1-15]

\subsection{graph signal clustering}
\lipsum[1-15]

\subsection{varargin}
\lipsum[1-15]

\section{conclusions}
\lipsum[1-15]

\section{acknoledgements}
\lipsum[1-15]

\end{document}
