\chapter{Abstract}
\label{ch:introduction}
Nowadays huge amounts of data are collected every day, from engineering sciences to our personal data regarding health monitoring, to financial data or political influences. The analysis and the process of these huge datasets has become a non indifferent challenge, since, in fact, data tends to reside on complex and irregular structures, which progressively required the introduction of innovative approaches to better handle and utilize them. \cite{Ortega2017} \cite{Sandry}

It is in this context that a few years ago the \gls{gsp} filed emerged.

\section{Work structure}
The main focus of our work was over a branch of this field that deals with graphs and graph signals' representations: in details we tried to enhance a certain category of algorithms designed for learning purposes.
To summarize, the main steps in which the work is articulated are the following:
\begin{itemize}
\item We present and describe the main concepts necessary to the comprehension of the work, such as the concepts of \textit{graph signal}, \textit{dictionary} and \textit{smoothness prior};
\item We give a brief excursus on the related work and we present the main algorithms we worked on for improvements;
\item We present the improvements we applied to the learning problems examined, both for the dictionary learning and for the graph learning part;
\item We explain in details the benefits these additions bring to the issue regarding learning graph signal structures and show the results;
\item We draw some conclusions and briefly present the possible next steps in this work, in order to obtain further implementations;
\end{itemize}
