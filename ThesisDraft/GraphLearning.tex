\chapter{The Graph learning issue}
For the most part, the research regarding graph signal processing has been focusing on working with the signals spanned onto a graph manifold, assuming the graph structure to be already known. However, the choice of the underlying graph for the signal not necessarily represents faithfully the ground truth intrinsic manifold over which the signal is structured. In these cases, a more suitable approach to the problem could be trying to learn the graph structure underneath, such that the processing of the signal can rely on assumptions that better capture the intrinsic relationships between the entities or make them clear when otherwise it would not have be so. \cite{Maretic2017} Clearly the question is not trivial, since in general the action of learning a graph from sample data represents an ill-posed problem, with the consequence that there may be many solutions to the structure associated to a data. \cite{Dong2016} To overcome this obstacle several approaches have been designed in the recent years, and to some of them we will give a more detailed description in the next chapters, when we will address the basis of our work.
